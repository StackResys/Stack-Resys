\documentclass{article} % For LaTeX2e
\usepackage{nips11submit_e,times}
%\documentstyle[nips10submit_09,times,art10]{article} % For LaTeX 2.09

\title{Predicting the Tags of Questions in StackOverflow}

\newcommand{\fix}{\marginpar{FIX}}
\newcommand{\new}{\marginpar{NEW}}

%\nipsfinalcopy % Uncomment for camera-ready version

\begin{document}

\maketitle
\section{Overview}
As the world's most popular programmer Q\&A community, StackOverflow {ref} is a showcase of the successful usage of the tag system, where each question could have one or more questions to indicate its "topics". As a more flexible way of categorizing the text, the tags also reveals rich semantic information about questions. In this project, we focused on the prediction the tags of new questions by mining the large amount of tagged questions.

Furthermore, a tags predictor of high accuracy also enables us to discover user's taste of questions. For example, by examing the questions a user asked/answered/favorited/voted, we can assign her the weights for different tags. The weight could therefore become an effective indicator of the user's interest. As a result, we can use this information to recommend new question or discover other user who share the similiar interests.

\section{Dataset}
StackOverflow has published their dataset under CC(Common Creative) Licence. The dataset includes over 2.2 millions questions, 4.8 million answers and 30 thousands of tags are created in StackOverflow.

\section{Investigation}
With the investigation of popular text classifcation methods, we found these approaches are promising in the processing of multi-labeled supervised text classification (1) Decision Tree (2) Naïve Bayes (3) Support Vector Machines. In our project we will adopt and improve these methods for the classification tasks.

\section{Project Roadmap}
Here's the roadmap of our project:
\begin{enumerate}
    \item Data Observation(Before mid-term): Oberserve and pre-processing the data.
    \item Investigation(Before mid-term): In-depth investigate several promising text classification methods.
    \item Feature Selection(Before mid-term): By analyzing the data, we will select the features are more likely to achieve good performance and high accuracy.
    \item Development of classification models(70\% done before mid-term)
    \item Experiments(After mid-term): In the experiments, we will focus on the performance analysis, feature analysis, error analysis and comparison of methods.
    \item Demo(optional): The demo will demonstrate the possible usage of our predicator, such as tags suggestion, qeustion recommendation, etc.
\end{enumerate}
\end{document}

