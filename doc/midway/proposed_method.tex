
\subsection{Naive Bayes}
Naive Bayes \cite{NaiveBayes} classifier is a simple yet powerful classifier based on applying Bayes theorem. It assumes that the presence of a feature is independent from the presence of the other features. When performing the text categorization, Naive Bayes treats each document as a "bag of words" and words are conditionally independent from each other.

In practice, naive Bayes classifiers can have a very satisfactory performance in a supervised learning. Many real-world problems are tackled by naive Bayes models uses the method of maximum likelihood; in other words, one can work with the naive Bayes model without believing in Bayesian probability or using any Bayesian methods.


How it works in our project

the Advantage.

the disadvantage.
\subsection{Logistic Regression}

\renewcommand{\arraystretch}{1}
\vspace{-0.02in}
\begin{table}[htb]
\centering
\small
\begin{tabular}{l} \hline
\textbf{Algorithm 2:} Logistic Regression\\ \hline
\textbf{Input:} Training set $T=\{t_1,...t_n\}$, constant $\eta$, converge threshold $\varepsilon$\\
%\hspace{0.65cm} Test set: Set of queries $Q$;\\
\textbf{Output:} Weight matrix $W$\\

\hspace{.15cm}1: Initialize all $w_{ij} \in W$ to 0;\\
\hspace{.15cm}2: $isConverge \leftarrow false$;\\
\hspace{.15cm}3: \textbf{While} $isConverge = false$;\\
\hspace{.15cm}4: \quad \textbf{Foreach} $w_{ij} \in W$\\
\hspace{.15cm}5: \quad \quad \textbf{Foreach} $t_l \in T$ \textbf{do} \\
\hspace{.15cm}6: \quad \quad \quad $jump_{ij} \leftarrow 0$;\\
\hspace{.15cm}7: \quad \quad \quad Calculate $d=X_{i}^{l}(\delta (Y^{l}=y_{j})-\hat{P}(Y^{l}=y_{j}|X^{l},W))$;\\
\hspace{.15cm}8: \quad \quad \quad $jump_{ij} \leftarrow \eta * d$; \\
\hspace{.15cm}9: \quad \quad \textbf{End}\\
10: \quad \quad $w_{ij} \leftarrow w_{ij} + jump_{ij}$;\\
11: \quad \textbf{End}\\
12: \quad \textbf{If} $\forall jump_{ij} \rightarrow jump_{ij} < \varepsilon$\\
13: \quad \quad $isConverge \leftarrow true$;\\
14: \quad \textbf{End}\\
15: \textbf{End}\\
16: \textbf{Return} $W$;\\
\hline
\end{tabular}
\normalsize
\end{table}

\subsection{SVM}
