
\subsection{Text Categorization}

As F. Sebastiani described in \cite{DBLP:journals/csur/Sebastiani02}, great effort has been made to automatically categorize text into predefined categories in recent years, and the current dominant approaches are based on machine learning techniques. By learning from preclassified documents, a classifier can grasp the characteristics of the categories and predict the classes of new documents. 
%Three fundamental problems in text categorization are document representation, classifier construction and classifier evaluation. 
As texts cannot be directly interpreted by a classifier, an indexing procedure which maps a text $d_j$ into a compact representation of its contents is called on all documents (training, validation and test documents). And dimension reduction techniques are also applied to the documents to reduce the size of vector space and can also contributes to reducing overfitting problem. Among the text classifiers most frequently used, decision tree, probabilistic classifiers (e.g. Naive Bayes), regression methods, neural networks, k-NN, Support Vector Machines can be applied to our problem.

\subsection{Social Tag Prediction}
Social tag prediction \cite{DBLP:conf/sigir/HeymannRG08} is a recent emerged problem in social networks. Given a set of objects (e.g. texts) and a set of tags applied to those objects by users, the problem is to predict whether a given tag can be applied to a particular object. Notice that an object can have be assigned with multiple tags. While standard taxonomies force objects into predefined categories, tags have no such limitations. The distribution of tags may change rapidly and little is known about the predictability of tags. By attacking this problem, we can use a tag predictor to enhance a social tagging system by increasing recall of a single tag, suggesting tags for users when they create new objects, sharing objects despite of vocabulary differences between users and helping designers to analyze tag usage.